
\documentclass{article}

\usepackage{fancyhdr} % Required for custom headers
\usepackage{lastpage} % Required to determine the last page for the footer
\usepackage{extramarks} % Required for headers and footers
\usepackage[usenames,dvipsnames]{color} % Required for custom colors
\usepackage{graphicx} % Required to insert images
\usepackage{listings} % Required for insertion of code
\usepackage{courier} % Required for the courier font
\usepackage{lipsum} % Used for inserting dummy 'Lorem ipsum' text into the template

% Margins
\topmargin=-0.45in
\evensidemargin=0in
\oddsidemargin=0in
\textwidth=6.5in
\textheight=9.0in
\headsep=0.25in

\linespread{1.1} % Line spacing

% Set up the header and footer
\pagestyle{fancy}
%\lhead{\hmwkAuthorName} % Top left header
\lhead{\hmwkClass\ (\hmwkClassInstructor\ ): \hmwkTitle} % Top center head
\rhead{\firstxmark} % Top right header
\lfoot{\lastxmark} % Bottom left footer
\cfoot{} % Bottom center footer
\rfoot{Page\ \thepage\ of\ \protect\pageref{LastPage}} % Bottom right footer
\renewcommand\headrulewidth{0.4pt} % Size of the header rule
\renewcommand\footrulewidth{0.4pt} % Size of the footer rule

\setlength\parindent{0pt} % Removes all indentation from paragraphs

%----------------------------------------------------------------------------------------
%	DOCUMENT STRUCTURE COMMANDS
%	Skip this unless you know what you're doing
%----------------------------------------------------------------------------------------

% Header and footer for when a page split occurs within a problem environment
\newcommand{\enterProblemHeader}[1]{
\nobreak\extramarks{#1}{#1 continued on next page\ldots}\nobreak
\nobreak\extramarks{#1 (continued)}{#1 continued on next page\ldots}\nobreak
}

% Header and footer for when a page split occurs between problem environments
\newcommand{\exitProblemHeader}[1]{
\nobreak\extramarks{#1 (continued)}{#1 continued on next page\ldots}\nobreak
\nobreak\extramarks{#1}{}\nobreak
}

\setcounter{secnumdepth}{0} % Removes default section numbers
\newcounter{homeworkProblemCounter} % Creates a counter to keep track of the number of problems

\newcommand{\homeworkProblemName}{}
\newenvironment{homeworkProblem}[1][Problem \arabic{homeworkProblemCounter}]{ % Makes a new environment called homeworkProblem which takes 1 argument (custom name) but the default is "Problem #"
\stepcounter{homeworkProblemCounter} % Increase counter for number of problems
\section{\homeworkProblemName} % Make a section in the document with the custom problem count
\enterProblemHeader{\homeworkProblemName} % Header and footer within the environment
}{
\exitProblemHeader{\homeworkProblemName} % Header and footer after the environment
}

\newcommand{\problemAnswer}[1]{ % Defines the problem answer command with the content as the only argument
\noindent\framebox[\columnwidth][c]{\begin{minipage}{0.98\columnwidth}#1\end{minipage}} % Makes the box around the problem answer and puts the content inside
}

\newcommand{\homeworkSectionName}{}
\newenvironment{homeworkSection}[1]{ % New environment for sections within homework problems, takes 1 argument - the name of the section
\renewcommand{\homeworkSectionName}{#1} % Assign \homeworkSectionName to the name of the section from the environment argument
\subsection{\homeworkSectionName} % Make a subsection with the custom name of the subsection
\enterProblemHeader{\homeworkProblemName\ [\homeworkSectionName]} % Header and footer within the environment
}{
\enterProblemHeader{\homeworkProblemName} % Header and footer after the environment
}

%----------------------------------------------------------------------------------------
%	NAME AND CLASS SECTION
%----------------------------------------------------------------------------------------

\newcommand{\hmwkTitle}{Assignment - R Language} % Assignment title
\newcommand{\hmwkDueDate}{Wednesday,\ February\ 10,\ 2016} % Due date
\newcommand{\hmwkClass}{ MA208} % Course/class
\newcommand{\hmwkClassInstructor}{Submitted to Dr. Vishwanath K. P} % Teacher/lecturer
\newcommand{\hmwkAuthorName}{Kishan Desai 13CO212 } % Your name

%----------------------------------------------------------------------------------------
%	TITLE PAGE
%----------------------------------------------------------------------------------------

\title{
\vspace{2in}
\textmd{\textbf{\hmwkClass:\ \hmwkTitle}}\\
\normalsize\vspace{0.1in}\small{Due\ on\ \hmwkDueDate}\\
\vspace{0.1in}\large{\textit{\hmwkClassInstructor}}
\vspace{3in}
}

\author{\textbf{\hmwkAuthorName}}
\date{10th Feb, 2016} % Insert date here if you want it to appear below your name

%----------------------------------------------------------------------------------------

\begin{document}

\maketitle

%----------------------------------------------------------------------------------------
%	TABLE OF CONTENTS
%----------------------------------------------------------------------------------------

%\setcounter{tocdepth}{1} % Uncomment this line if you don't want subsections listed in the ToC

\newpage
\tableofcontents
\newpage


\section{Introduction}
R is a software language for carrying out complicated (and simple) statistical analyses. It includes
routines for data summary and exploration, graphical presentation and data modelling. It is a GNU project which is similar to the S language and environment which was developed at Bell Laboratories  by John Chambers and colleagues. R can be considered as a different implementation of S.\\
\\
R provides a wide variety of statistical (linear and nonlinear modelling, classical statistical tests, time-series analysis, classification, clustering, …) and graphical techniques, and is highly extensible. The S language is often the vehicle of choice for research in statistical methodology, and R provides an Open Source route to participation in that activity.\\

One of R’s strengths is the ease with which well-designed publication-quality plots can be produced, including mathematical symbols and formulae where needed. Great care has been taken over the defaults for the minor design choices in graphics, but the user retains full control.

\subsection{Setting up R in Windows}
For Setting up R in windows- go to url: https://cran.r-project.org/bin/windows/base/. Download the software package for R. Then install the file in windows using install manager. Open R compiler and start screen will look like this:\\
\begin{figure}[h!]
\includegraphics[scale=0.5]{r.png}
  \centering
  \caption{R window}
  \label{fig:barchart1}
\end{figure}

 
\subsection{R-basics}
R stores information and operates on objects. The simplest objects are scalars, vectors and matrices.
But there are many others: lists and dataframes for example. In advanced use of R it can also be
useful to define new types of object, specific for particular application.\\

Sample arithmetic can be done as:

\begin{lstlisting}[frame=single]
> 4+5
[1] 9
\end{lstlisting}
 
We can assign objects values for subsequent use. For example:
\begin{lstlisting}[frame=single]
> x = 6
> y = 4
> z = x+y
> z
[1] 10
\end{lstlisting}

At any time we can list the objects which we have created:

\begin{lstlisting}[frame=single]
> ls()
[1] "x" "y" "z"

\end{lstlisting}

There are many standard functions available in R, and it is also possible to create new ones.
Vectors can be created in R in a number of ways. We can describe all of the elements:
\begin{lstlisting}[frame=single]
> vec = c(5,9,1,0)
> vec
[1] 5 9 1 0
\end{lstlisting}

\begin{itemize}
\item There are many inbuilt functions in R for statistical analysis. 
\item Such as mean(), median() and var() are some of the popular measures. 
\end{itemize}

So for our example:

\begin{lstlisting}[frame=single]
> mean(vec)
[1] 3.75
> median(vec)
[1] 3
> var(vec)
[1] 16.91667
\end{lstlisting}

We can make different instances of same data type. (\# is used for comments)
\begin{lstlisting}[frame=single]
> vec.draft1 = c(5,9,0,1)
> vec.draft2 = typos.draft1 	# make a copy
> vec.draft2[1] = 0 		# assign the first page 0 typos
\end{lstlisting}


R makes it easy to translate mathematics in a natural way once your data is read in. We can define user defined functions in R as follow:
\begin{lstlisting}[frame=single]
> arr = c(100, 158, 75, 69, 104, 110, 115, 112)
> max(arr)
[1] 158
> fun = function(x) sqrt(var(x))
> fun(arr)
[1] 27.26556
\end{lstlisting}



\section{Plotting Functions}

\subsection{Bar Charts}
A bar chart draws a bar with a a height proportional to the count in the table. The height could be given by the
frequency, or the proportion. The graph will look the same, but the scales may be different.

Sample bar chart is given:

\begin{lstlisting}[frame=single ]
> var = c(1,2,1,4,1,5,4,1,2,5,4,4,3,1)
> barplot(table(var))
\end{lstlisting} 

So we will get bar chart like this:\\

 
\begin{figure}[h!]
\includegraphics[scale=0.4]{bar.png}
  \centering
  \caption{Bar chart}
  \label{fig:barchart1}
\end{figure}


\subsection{Pie Charts}
The same data can be studied with pie charts using the pie function. We use the same data as above:

\begin{lstlisting}[frame=single ]
> var = c(1,2,1,4,1,5,4,1,2,5,4,4,3,1)
>  pie(beer.counts,col=c("purple","green2","cyan","white"))
\end{lstlisting} 

So we will get pie chart like this:\\

 
\begin{figure}[h!]
\includegraphics[scale=0.4]{pie.png}
  \centering
  \caption{Pie chart}
  \label{fig:barchart1}
\end{figure}


\subsection{Numerical data}
To describe a distribution we often want to know where is it centered and what is the spread. These are typically
measured with mean and variance (or standard deviation), or the median and more generally the five-number summary.
The R commands for these are mean, var, sd, median, fivenum and summary. Summary command gives mean, median, max, mean, 25 and 75 quantile of the data. ( The p quantile, (also known as the 100p%-percentile) is the point
in the data where 100p\% is less, and 100(1-p)\% is larger )

\begin{lstlisting}[frame=single ]
> data = c(15,20,26,14,25,39,12)
> summary(data)
    Min. 1st Qu.  Median    Mean 3rd Qu.    Max. 
   12.00   14.50   20.00   21.57   25.50   39.00
\end{lstlisting}

\subsection{Histogram}

The histogram defines a sequence of breaks and then counts the number of observation in the bins formed by the breaks.  It plots these with a bar similar to the bar chart, but the bars are touching. The height can
be the frequencies, or the proportions.

Sample program look like:
\begin{lstlisting}[frame=single ]
> var =c(2,5,2,1,4,3,3,6,2,1,2,4,4,6,6,6,1,2,4,2,3,5,6)
> hist(var)
\end{lstlisting}	

So we will get histogram like this:\\

 
\begin{figure}[h!]
\includegraphics[scale=0.4]{hist.png}
  \centering
  \caption{Pie chart}
  \label{fig:barchart1}
\end{figure}

\section{Bivariate data}
Bivariate data is used to combine two variables and it is summarized using table. We can use bivariate data by creating two independent vectors and then combining them using table command.

\begin{lstlisting}[frame=single ]
> car = c("Maruti","Hyundai","Maruti","Maruti","Hyundai","Maruti","Hyundai","Maruti","Hyundai","Maruti")
> amount = c(2,1,1,2,1,3,3,2,3,1)
> table(car,amount)
         amount
car       1 2 3
  Hyundai 2 0 2
  Maruti  2 3 1
> barplot(table(car,amount),col=c("purple","green2"),
+ beside=TRUE)

\end{lstlisting}

\begin{figure}[h!]
\includegraphics[scale=0.4]{table.png}
  \centering
  \caption{Table chart}
  \label{fig:barchart1}
\end{figure}


\section{Conclusion}

The R programming language is an important tool for development in the numeric analysis and machine learning spaces. With machines becoming more important as data generators, the popularity of the language can only be expected to grow.\\

The main advantages of R language are:\\
\begin{itemize}
\item R is a powerful scripting language
\item Graphics and data visualization
\item Integration with document publishing
\item Access to powerful, cutting-edge analytics
\item No cost

\end{itemize}

The R programming language is mainly used by data scientists and statisticians to extract or data mine information from a large data set or surveys.
%----------------------------------------------------------------------------------------

\end{document}